%% Based on a TeXnicCenter-Template by Gyorgy SZEIDL.
%%%%%%%%%%%%%%%%%%%%%%%%%%%%%%%%%%%%%%%%%%%%%%%%%%%%%%%%%%%%%

%------------------------------------------------------------
%
\documentclass{article}%
%Options -- Point size:  10pt (default), 11pt, 12pt
%        -- Paper size:  letterpaper (default), a4paper, a5paper, b5paper
%                        legalpaper, executivepaper
%        -- Orientation  (portrait is the default)
%                        landscape
%        -- Print size:  oneside (default), twoside
%        -- Quality      final(default), draft
%        -- Title page   notitlepage, titlepage(default)
%        -- Columns      onecolumn(default), twocolumn
%        -- Equation numbering (equation numbers on the right is the default)
%                        leqno
%        -- Displayed equations (centered is the default)
%                        fleqn (equations start at the same distance from the right side)
%        -- Open bibliography style (closed is the default)
%                        openbib
% For instance the command
%           \documentclass[a4paper,12pt,leqno]{article}
% ensures that the paper size is a4, the fonts are typeset at the size 12p
% and the equation numbers are on the left side
%
\usepackage{amsmath}%
\usepackage{amsfonts}%
\usepackage{amssymb}%
\usepackage{graphicx}
\usepackage[utf8]{inputenc}
\usepackage[T1]{fontenc}
\usepackage{lmodern}
%-------------------------------------------
\newtheorem{theorem}{Theorem}
\newtheorem{acknowledgement}[theorem]{Acknowledgement}
\newtheorem{algorithm}[theorem]{Algorithm}
\newtheorem{axiom}[theorem]{Axiom}
\newtheorem{case}[theorem]{Case}
\newtheorem{claim}[theorem]{Claim}
\newtheorem{conclusion}[theorem]{Conclusion}
\newtheorem{condition}[theorem]{Condition}
\newtheorem{conjecture}[theorem]{Conjecture}
\newtheorem{corollary}[theorem]{Corollary}
\newtheorem{criterion}[theorem]{Criterion}
\newtheorem{definition}[theorem]{Definition}
\newtheorem{example}[theorem]{Example}
\newtheorem{exercise}[theorem]{Exercise}
\newtheorem{lemma}[theorem]{Lemma}
\newtheorem{notation}[theorem]{Notation}
\newtheorem{problem}[theorem]{Problem}
\newtheorem{proposition}[theorem]{Proposition}
\newtheorem{remark}[theorem]{Remark}
\newtheorem{solution}[theorem]{Solution}
\newtheorem{summary}[theorem]{Summary}
\newenvironment{proof}[1][Proof]{\textbf{#1.} }{\ \rule{0.5em}{0.5em}}

\begin{document}

\title{A Genetic Algorithm for estimating physicochemical properties to achieve desired drug release}

\author{
	Cláudio César F. A. Almeida
	\\\email{ccesar@dcc.ufmg.br}
	\\Departamento de Ciência da Computação-ICEx
	\\Universidade Federal de Minas Gerais, Brasil
	\and
	\\José Lopes de Siqueira Neto
	\\\email{jose@dcc.ufmg.br}
	\\Departamento de Ciência da Computação-ICEx
	\\Universidade Federal de Minas Gerais, Brasil
	\and
	\\Jadson C. Belchior
	\\\email{jadson@ufmg.br}
	\\Departamento de Química-ICEx,
	\\Universidade Federal de Minas Gerais, Brasil
	\and
	\\Domingos D. C. Rodrigues
	\\\email{xxxxx@ufmg.br}
	\\Laboratório de Computação Científica-ICEx,
	\\Universidade Federal de Minas Gerais, Brasil
}

\date{February 31, 2015}
\maketitle

\begin{abstract}
\\\cite{ahu61}
\\
An alternative methodology based on genetic algorithm is proposed to be a complementary tool to other conventional methods to study controlled drug release. Two systems are used to test the approach; namely, hydrocortisone in a biodegradable matrix and rhodium (II) butyrate complexes in a bioceramic matrix. Two well-established mathematical models are used to simulate different release profiles
as a function of fundamental properties; namely, diffusion coefficient (D), saturation
solubility (Cs), drug loading (A), and the height of the device (h). The models were tested,
and the results show that a wide range of values for these fundamental properties can be predicted to achieve the desired drug release. The genetic algorithm results obtained can be considered to quantitatively predict ideal experimental conditions. Overall, the proposed methodology was shown to be efficient for ideal experiments, with a relative average error of <1\% in both tests. This approach can be useful for the experimental analysis to simulate and design efficient controlled drug release systems.
\\
\\\textbf{keywords:} genetic algorithm; drug release; controlled release; mathematical models
\end{abstract}

\section{Introduction}

Over recent years there has been a growing interest in drug delivery studies from solid
pharmaceutical dosage forms. Drug delivery systems (DDS hereinafter) must be developed
taking into account the economical aspects involved in the industrial mass production, ZzZzZthat
is to say that DDS must be manufactured at low cost without compromising the therapeutic
goal of the drug. This has prompted the pharmaceutical industry to search for new
cheap raw materials for the manufacture of delivery systems and for the formulation of
new encapsulation techniques ZzZzZ([13] and references therein). ZzZzZIt is therefore important to be
able to quantify the dependence of the transport mechanisms involved in the release of any
drug from the physicochemical properties of the carrier material. In general the therapeutic
treatments consist in delivering the drug to some specific region or ZzZzZorgan of the body which
therefore requires that the medication must be released in a controlled and sustained form
to keep its concentration in the body at accepted levels. Due to ease of production and
low cost associated, the usual procedure is to disperse some drug in a hydrophobic (ZzZzZe.g.) or
hydrophilic (e.g. hydroxypropyl methylcellulose-HPMC) biomaterial ZzZzZso that a controlled
release of the drug into the biological medium can be ZzZzZattained. In the pharmaceutical literature
this combination of drug and ZzZzZexcipient agent is commonly referred as matrix. Overall,
the release of any drug from some matrix (consisting of bioceramics or biodegradable polymers)
involves both the physical phenomena of dissolution - or erosion - and diffusion. Once
the tablet carrying the drug in suspension enters in contact with water or the physiological
fluid, the latter diffuses into the tablet (ZzZzZimbibition) increasing its volume (ZzZzZswelling) and
promoting the drug dissolution. The dissolved drug will then diffuse out of the device due
to steep concentration gradients. For water insoluble polymers, the diffusional mass transfer
is preceded by cleavage of the polymer chains that constitute the matrix (erosion).
Several phenomenological models have been proposed with various degrees of sophistication
to describe such processes in simple and compact mathematical forms (for a comprehensive
review see ZzZzZ[2]), but full treatment of the involved transport mechanisms often need to
be done numerically. When no matrix erosion or swelling occurs (or they are instantaneous)
and drug molecules do not interact, the release is basically controlled by the mechanism of
Fickian diffusion which is a well understood process with a very precise mathematical formulation
for any geometry of the device and deeply rooted in physical first principles (ZzZzZ[3]).
\\
\\Even so, simple analytical solutions are only obtained when one imposes strict assumptions,
as constant drug diffusivity, perfect sink conditions, fixed boundaries and an infinite external
medium. When dissolution or erosion becomes relevant, the resulting mathematical
equations can get somewhat cumbersome as the drug diffusion in a substrate can depend
on various parameters, such as the matrix composition, matrix geometry, volume expansion
(swelling), erosion, polymer dissolution, initial drug loading and drug saturation solubility
in the matrix ZzZzZ[13, 15]. Therefore depending on the degree of simplification of the kinetics
behind drug release, one can get an explicit or implicit analytical relation between the independent
(mass-fraction of drug release and time) and dependent (physicochemical properties
of the matrix) variables of the problem.
From the technological point of view, explicit analytical ZzZzZformulae as simple as possible are
mostly desired in order to simulate and access the economical viability of alternative paths
of the manufacturing of DDS in a rapid and efficient manner, as well as for routine quality
control processes. Several statistical methods have been proposed to test the similarity between
dissolution profiles of test and reference batches as well as to decide the best model to
study the drug dissolution ( we refer the reader to the review of ZzZzZCosta and Sousa Lobo [2]).
Given the elevated number of variables involved, ZzZzZone does in practice several cuts in the
parameter space and performs a nonlinear regression analysis to fit the experimental data
to some chosen model ZzZzZ[5]. Quite recently a complementary approach to these conventional
methods based on artificial neural networks (ANN) has been used (see review of Sun et al.
ZzZzZ[17] and references therein). This approach has proved to be useful in predicting ideal experimental
parameters for providing a controlled drug release. It uses the whole experimental
data training set and correlates it with the input parameters of the chosen formulation by
optimizing the weights and biases of the ANN through some minimization technique. There
is no need to have a priori functional relationship between the dependent and independent
variables although in practice ZzZzZone uses some accepted model to simulate a great number of
experiments to increase the training set. After the training stage, one can quickly predict
new formulation input parameters (geometry and physicochemical properties of the matrix)
for any dissolution profile of any new submitted experimental batch (assuming the experimental
conditions have been maintained as those of the training set). The use of ANNs
is nevertheless limited, as they cannot provide any information at all about the underlying
mechanisms of drug dissolution and diffusion. It was in this context that ZzZzZReis et al. [12]


\noindent The front matter has various entries such as\\
\hspace*{\fill}\verb" \title", \verb"\author", \verb"\date", and
\verb"\thanks"\hspace*{\fill}\\
You should replace their arguments with your own.

This text is the body of your article. You may delete everything between the commands\\
\hspace*{\fill} \verb"\begin{document}" \ldots \verb"\end{document}"
\hspace*{\fill}\\in this file to start with a blank document.


\section{The Most Important Features}

\noindent Sectioning commands. The first one is the\\
\hspace*{\fill} \verb"\section{The Most Important Features}" \hspace*{\fill}\\
command. Below you shall find examples for further sectioning commands:

\subsection{Subsection}
Subsection text.

\subsubsection{Subsubsection}
Subsubsection text.

\paragraph{Paragraph}
Paragraph text.

\subparagraph{Subparagraph}Subparagraph text.\vspace{2mm}

Select a part of the text then click on the button Emphasize (H!), or Bold (Fs), or
Italic (Kt), or Slanted (Kt) to typeset \emph{Emphasize}, \textbf{Bold},
\textit{Italics}, \textsl{Slanted} texts.

You can also typeset \textrm{Roman}, \textsf{Sans Serif}, \textsc{Small Caps}, and
\texttt{Typewriter} texts.

You can also apply the special, mathematics only commands $\mathbb{BLACKBOARD}$
$\mathbb{BOLD}$, $\mathcal{CALLIGRAPHIC}$, and $\mathfrak{fraktur}$. Note that
blackboard bold and calligraphic are correct only when applied to uppercase letters A
through Z.

You can apply the size tags -- Format menu, Font size submenu -- {\tiny tiny},
{\scriptsize scriptsize}, {\footnotesize footnotesize}, {\small small}, {\normalsize
normalsize}, {\large large}, {\Large Large}, {\LARGE LARGE}, {\huge huge} and {\Huge
Huge}.

You can use the \verb"\begin{quote} etc. \end{quote}" environment for typesetting
short quotations. Select the text then click on Insert, Quotations, Short Quotations:

\begin{quote}
The buck stops here. \emph{Harry Truman}

Ask not what your country can do for you; ask what you can do for your
country. \emph{John F Kennedy}

I am not a crook. \emph{Richard Nixon}

I did not have sexual relations with that woman, Miss Lewinsky. \emph{Bill Clinton}
\end{quote}

The Quotation environment is used for quotations of more than one paragraph. Following
is the beginning of \emph{The Jungle Books} by Rudyard Kipling. (You should select
the text first then click on Insert, Quotations, Quotation):

\begin{quotation}
It was seven o'clock of a very warm evening in the Seeonee Hills when Father Wolf woke
up from his day's rest, scratched himself, yawned  and spread out his paws one after
the other to get rid of sleepy feeling in their tips. Mother Wolf lay with her big gray
nose dropped across her four tumbling, squealing cubs, and the moon shone into the
mouth of the cave where they all lived. ``\emph{Augrh}'' said Father Wolf, ``it is time
to hunt again.'' And he was going to spring down hill when a little shadow with a bushy
tail crossed the threshold and whined: ``Good luck go with you, O Chief of the Wolves;
and good luck and strong white teeth go with the noble children, that they may never
forget the hungry in this world.''

It was the jackal---Tabaqui the Dish-licker---and the wolves of India despise Tabaqui
because he runs about making mischief, and telling tales, and eating rags and pieces of
leather from the village rubbish-heaps. But they are afraid of him too, because
Tabaqui, more than any one else in the jungle, is apt to go mad, and then he forgets
that he was afraid of anyone, and runs through the forest biting everything in his way.
\end{quotation}

Use the Verbatim environment if you want \LaTeX\ to preserve spacing, perhaps when
including a fragment from a program such as:
\begin{verbatim}
#include <iostream>         // < > is used for standard libraries.
void main(void)             // ''main'' method always called first.
{
 cout << ''This is a message.'';
                            // Send to output stream.
}
\end{verbatim}
(After selecting the text click on Insert, Code Environments, Code.)


\subsection{Mathematics and Text}

It holds \cite{KarelRektorys} the following
\begin{theorem}
(The Currant minimax principle.) Let $T$ be completely continuous selfadjoint operator
in a Hilbert space $H$. Let $n$ be an arbitrary integer and let $u_1,\ldots,u_{n-1}$ be
an arbitrary system of $n-1$ linearly independent elements of $H$. Denote
\begin{equation}
\max_{\substack{v\in H, v\neq
0\\(v,u_1)=0,\ldots,(v,u_n)=0}}\frac{(Tv,v)}{(v,v)}=m(u_1,\ldots, u_{n-1})
\label{eqn10}
\end{equation}
Then the $n$-th eigenvalue of $T$ is equal to the minimum of these maxima, when
minimizing over all linearly independent systems $u_1,\ldots u_{n-1}$ in $H$,
\begin{equation}
\mu_n = \min_{\substack{u_1,\ldots, u_{n-1}\in H}} m(u_1,\ldots, u_{n-1}) \label{eqn20}
\end{equation}
\end{theorem}
The above equations are automatically numbered as equation (\ref{eqn10}) and
(\ref{eqn20}).

\subsection{List Environments}

You can create numbered, bulleted, and description lists using the tag popup
at the bottom left of the screen.

\begin{enumerate}
\item List item 1

\item List item 2

\begin{enumerate}
\item A list item under a list item.

The typeset style for this level is different than the screen style. \ The
screen shows a lower case alphabetic character followed by a period while the
typeset style uses a lower case alphabetic character surrounded by parentheses.

\item Just another list item under a list item.

\begin{enumerate}
\item Third level list item under a list item.

\begin{enumerate}
\item Fourth and final level of list items allowed.
\end{enumerate}
\end{enumerate}
\end{enumerate}
\end{enumerate}

\begin{itemize}
\item Bullet item 1

\item Bullet item 2

\begin{itemize}
\item Second level bullet item.

\begin{itemize}
\item Third level bullet item.

\begin{itemize}
\item Fourth (and final) level bullet item.
\end{itemize}
\end{itemize}
\end{itemize}
\end{itemize}

\begin{description}
\item[Description List] Each description list item has a term followed by the
description of that term. Double click the term box to enter the term, or to
change it.

\item[Bunyip] Mythical beast of Australian Aboriginal legends.
\end{description}

\subsection{Theorem-like Environments}

The following theorem-like environments (in alphabetical order) are available
in this style.

\begin{acknowledgement}
This is an acknowledgement
\end{acknowledgement}

\begin{algorithm}
This is an algorithm
\end{algorithm}

\begin{axiom}
This is an axiom
\end{axiom}

\begin{case}
This is a case
\end{case}

\begin{claim}
This is a claim
\end{claim}

\begin{conclusion}
This is a conclusion
\end{conclusion}

\begin{condition}
This is a condition
\end{condition}

\begin{conjecture}
This is a conjecture
\end{conjecture}

\begin{corollary}
This is a corollary
\end{corollary}

\begin{criterion}
This is a criterion
\end{criterion}

\begin{definition}
This is a definition
\end{definition}

\begin{example}
This is an example
\end{example}

\begin{exercise}
This is an exercise
\end{exercise}

\begin{lemma}
This is a lemma
\end{lemma}

\begin{proof}
This is the proof of the lemma.
\end{proof}

\begin{notation}
This is notation
\end{notation}

\begin{problem}
This is a problem
\end{problem}

\begin{proposition}
This is a proposition
\end{proposition}

\begin{remark}
This is a remark
\end{remark}

\begin{solution}
This is a solution
\end{solution}

\begin{summary}
This is a summary
\end{summary}

\begin{theorem}
This is a theorem
\end{theorem}

\begin{proof}
[Proof of the Main Theorem]This is the proof.
\end{proof}
\medskip

This text is a sample for a short bibliography. You can cite a book by making use of
the command \verb"\cite{KarelRektorys}": \cite{KarelRektorys}. Papers can be cited
similarly: \cite{Bertoti97}. If you want multiple citations to appear in a single set
of square brackets you must type all of the citation keys inside a single citation,
separating each with a comma. Here is an example: \cite{Bertoti97, Szeidl2001,
Carlson67}.

\begin{thebibliography}{9}


                                                                                                %
\bibitem {KarelRektorys}Rektorys, K., \textit{Variational methods in Mathematics,
Science and Engineering}, D. Reidel Publishing Company,
Dordrecht-Hollanf/Boston-U.S.A., 2th edition, 1975

\bibitem {Bertoti97} \textsc{Bert\'{o}ti, E.}:\ \textit{On mixed variational formulation
of linear elasticity using nonsymmetric stresses and displacements}, International
Journal for Numerical Methods in Engineering., \textbf{42}, (1997), 561-578.

\bibitem {Szeidl2001} \textsc{Szeidl, G.}:\ \textit{Boundary integral equations for
plane problems in terms of stress functions of order one}, Journal of Computational and
Applied Mechanics, \textbf{2}(2), (2001), 237-261.

\bibitem {Carlson67}  \textsc{Carlson D. E.}:\ \textit{On G\"{u}nther's stress functions
for couple stresses}, Quart. Appl. Math., \textbf{25}, (1967), 139-146.
\end{thebibliography}

\bibliographystyle{te}

\bibliography{Bibliografia}


\appendix

\section{The First Appendix}

The appendix fragment is used only once. Subsequent appendices can be created
using the Section Section/Body Tag.
\end{document}
